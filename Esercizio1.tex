\documentclass{article}
\usepackage[round,sort]{natbib} % make the in-text citations with a round bracket, and sort by year
\usepackage{authblk} % allows pretty formatting of author lists
\usepackage[capitalise]{cleveref}  % when you reference a table or figure, automatically format as "Table 1" rather than "table 1" or "1".
\title{Esercizio 1 - Analisi dei furti di automobili a Chicago}
\author[1,2]{Uberto Vittorio Favero}
\author[1,3]{Ludovico Loreti}
\affil[1]{Università di Bologna}
\affil[2]{Statistica Numerica}

\usepackage[italian]{babel} 
\usepackage[utf8x]{inputenc}
\usepackage{amssymb}
\usepackage{amsmath}
\usepackage{Sweave}


\begin{document}
\input{Esercizio1-concordance}

\maketitle

\section{Introduzione}

In questa relazione esamineremo il file reso disponibile dal portale della città di Chicago dal 2001 inerente i furti di autoveicoli
così da descrivere il campione di dati, analizzarlo formandone un modello.

Il datasource è liberamente scaricabile a:
https://data.cityofchicago.org/Public-Safety/Crimes-2001-to-present/ijzp-q8t2

Ed è composto da:

\begin{itemize}
\item Data e ora di rilievo del reato
\item Indirizzo
\item Tipologia di reato
\item Descrizione del tipo di veicolo
\item Descrizione del luogo del furto (Strada, parcheggio privato, supermercato etc)
\item Ward, ovvero il distretto di polizia di riferimento
\item Community Area, ovvero il quartiere
\item Anno
\item Latitudine e longutudine
\end{itemize}
\texttt{kruskal.test} help page into a \LaTeX{} document:

\newpage

\section{Preparazione dei dati}

Il dati sono disponibili per il download in un file .csv di circa 200k record.

Innanzitutto valuto il file con la funzione scan per comprenderene la conformazione:

\begin{Schunk}
\begin{Sinput}
> library(ggplot2)
> library(stringr)
> library(data.table)
> library(ggmap)
> library(dplyr)
> #~/Desktop/UNIBO/Statistica /1_2017/dataset/motor_vehicle_theft.csv
> datiChicago <- read.csv('~/Documents/EsameStatistica2k17/ProbabilityAndStatisticsUsingR/dataset/motor_vehicle_theft.csv', 
+                         stringsAsFactors = FALSE)
> str(datiChicago)
\end{Sinput}
\begin{Soutput}
'data.frame':	295225 obs. of  9 variables:
 $ Date                : chr  "03/20/2006 11:00:00 PM" "03/15/2006 03:00:00 PM" "03/22/2006 12:30:00 AM" "03/20/2006 11:20:00 PM" ...
 $ Block               : chr  "064XX S WOLCOTT AVE" "092XX S LANGLEY AVE" "0000X E 118TH ST" "070XX S SOUTH CHICAGO AVE" ...
 $ Primary.Type        : chr  "MOTOR VEHICLE THEFT" "MOTOR VEHICLE THEFT" "MOTOR VEHICLE THEFT" "MOTOR VEHICLE THEFT" ...
 $ Description         : chr  "AUTOMOBILE" "AUTOMOBILE" "THEFT/RECOVERY: AUTOMOBILE" "AUTOMOBILE" ...
 $ Location.Description: chr  "STREET" "STREET" "STREET" "STREET" ...
 $ Ward                : int  15 9 9 6 2 43 47 15 4 15 ...
 $ Community.Area      : int  67 44 53 69 28 8 3 67 38 67 ...
 $ Year                : int  2006 2006 2006 2006 2006 2005 2006 2006 2006 2006 ...
 $ Location            : chr  "(41.776711023, -87.671429342)" "(41.72632605, -87.606999811)" "(41.679864218, -87.621850413)" "(41.767501554, -87.607862513)" ...
\end{Soutput}
\begin{Sinput}
> 
\end{Sinput}
\end{Schunk}
 
Ora che ne ho verificato la conformazione posso passare alla valutazione di eventuali dati mancanti
\begin{Schunk}
\begin{Sinput}
> 
> 
> 
\end{Sinput}
\end{Schunk}
% \begin{center}
% <<fig=TRUE,echo=TRUE>>=
% boxplot(Ozone ~ Month, data = airquality)
% @
% \end{center}



\section{ Analisi quantitativa dei dati}

Al fine di fare un'analisi quantitativa dei dati, decido di guardare in generale l'andamento dei furti in Chicago per anno.
Quindi appoggiandomi al pacchetto \texttt{dplyr} raggruppo il file per anno e calcolo i totali:

\begin{Schunk}
\begin{Sinput}
> totAnnuo <- group_by(datiChicago, Year, Description)
> totsPerCateg <- summarize(totAnnuo,count=n())
> str(totsPerCateg)
\end{Sinput}
\begin{Soutput}
Classes ‘grouped_df’, ‘tbl_df’, ‘tbl’ and 'data.frame':	174 obs. of  3 variables:
 $ Year       : int  2001 2001 2001 2001 2001 2001 2001 2001 2001 2002 ...
 $ Description: chr  "ATT: AUTOMOBILE" "ATT: TRUCK, BUS, MOTOR HOME" "ATTEMPT: CYCLE, SCOOTER, BIKE W-VIN" "AUTOMOBILE" ...
 $ count      : int  1439 56 3 20692 220 3228 11 274 1626 1092 ...
 - attr(*, "vars")=List of 1
  ..$ : symbol Year
 - attr(*, "drop")= logi TRUE
\end{Soutput}
\begin{Sinput}
> #aggrego il dataframe precedente per anno sommandone le categorie - 
> totPerAnno <- aggregate(totsPerCateg$count, by = list(Year=totsPerCateg$Year), FUN = sum)
> #rinomino colonna 
> names(totPerAnno)[names(totPerAnno) == 'x'] <- 'Totale.annuo'
> # xlab - ylab = labels. type = b both (linee e punti)
> 
> plot(totPerAnno, ylab = "Tot Annuo", xlab = "Anni", 
+      main = "Furti totali di veicoli a Chicago dal 2001 a oggi ", type = 'b')
> print(summary(totPerAnno))
\end{Sinput}
\begin{Soutput}
      Year       Totale.annuo  
 Min.   :2001   Min.   :  910  
 1st Qu.:2005   1st Qu.:12582  
 Median :2009   Median :18881  
 Mean   :2009   Mean   :17366  
 3rd Qu.:2013   3rd Qu.:22497  
 Max.   :2017   Max.   :27549  
\end{Soutput}
\end{Schunk}
Analizzando i dati appena stampanti si nota come ci sia un generale calo dei furti. Il dato che fa sorridere è il picco del 2017. Picco comprensibile in quanto l'anno non è ancora terminato. 

Sarebbe interessante nei paragrafi successivi stimare i furti di veicoli per quest'anno.

A questo punto il 2017 lo tratto come un outlier e per ora non lo prendo in considerazione, aggiornando i dataset rimuovendo i dati inerenti il 2017, e ricalcolo il \texttt{summary}, visualizzando anche meduana, massimo e minimo
\begin{Schunk}
\begin{Sinput}
> totsPerCateg <- totsPerCateg[!(totsPerCateg$Year == '2017'),]
> totPerAnno <- totPerAnno[!(totPerAnno$Year == '2017'),]
> knitr::kable(totPerAnno)
\end{Sinput}
\begin{Soutput}
| Year| Totale.annuo|
|----:|------------:|
| 2001|        27549|
| 2002|        25121|
| 2003|        22748|
| 2004|        22805|
| 2005|        22497|
| 2006|        21818|
| 2007|        18573|
| 2008|        18881|
| 2009|        15482|
| 2010|        19028|
| 2011|        19387|
| 2012|        16492|
| 2013|        12582|
| 2014|         9913|
| 2015|        10076|
| 2016|        11363|
\end{Soutput}
\begin{Sinput}
> plot(totPerAnno, ylab = "Tot Annuo", xlab = "Anni", 
+      main = "Furti totali di veicoli a Chicago dal 2001 a oggi ", type = 'b')
> 
> 
\end{Sinput}
\end{Schunk}
Ora, per studiare i dati in maniera più consona, calcolo le somme mensili per ogni tipologia di furto.
Ma per fare questo per evitare errori sui formati di data, converto la colonna \texttt{Date} secondo lo standard POSIXct - mentre ora sono salvati in POSIXlt.
Successivamente creo una tabella con le righe contenenti sulla prima colonna una sequenza di numeri dal 2001 al 2017 ordinati in maniera crescente divisi separati da un intervallo di 1/12 e calcolo per ogni mese i totali per ciascuno delle tipologie di furto indicate.
Per terminare salvo il csv così da poterlo riusare successivamente
\begin{Schunk}
\begin{Sinput}
> Sys.setenv(TZ='America/Sao_Paulo')
> #tz = timezone
> datiChicago$Date <- as.POSIXct(datiChicago$Date, tz="GMT", format="%m/%d/%Y %I:%M:%S %p")
> #creo una tabella vuota dove la prima colonna sono gli anni dal 2006 a oggi in forma numerica 2006, 2006.08333-...
> monthlyByCateg <- data.frame(Months =seq(as.Date("2001-01-01"), 
+                                          as.Date("2017-01-01"), by="months" ))
> #monthlyByCateg <- data.frame(Months = seq(from = 2001, to = 2017, by= 1/12 ))
> descrizioni <- as.character(unique(unlist(datiChicago$Description)))
> #genero la matrice con dati vuoti
> monthlyByCateg[,descrizioni] <-NA
> anno = 2001
> mese = 1
> rowsTot = nrow(monthlyByCateg)
> categTot <- length(descrizioni)
> #descrizionhe vale come indice colonna per la matrice monthlyByCateg
> 
> for(riga in 1: rowsTot){
+     #so fare due nested for, ma mi serviva così. ci mette 3 minuti buoni a calcolare tutto
+     monthlyByCateg[riga, 2]  <- summarize(subset(datiChicago, Year == anno 
+                                                  & Description == descrizioni[1] & month(Date) == mese ),count = n())
+     monthlyByCateg[riga, 3]  <- summarize(subset(datiChicago, Year == anno 
+                                                  & Description == descrizioni[2] & month(Date) == mese ),count = n())
+     monthlyByCateg[riga, 4]  <- summarize(subset(datiChicago, Year == anno 
+                                                  & Description == descrizioni[3] & month(Date) == mese ),count = n())
+     monthlyByCateg[riga, 5]  <- summarize(subset(datiChicago, Year == anno 
+                                                  & Description == descrizioni[4] & month(Date) == mese ),count = n())
+     monthlyByCateg[riga, 6]  <- summarize(subset(datiChicago, Year == anno 
+                                                  & Description == descrizioni[5] & month(Date) == mese ),count = n())
+     monthlyByCateg[riga, 7]  <- summarize(subset(datiChicago, Year == anno 
+                                                  & Description == descrizioni[6] & month(Date) == mese ),count = n())
+     monthlyByCateg[riga, 8]  <- summarize(subset(datiChicago, Year == anno 
+                                                  & Description == descrizioni[7] & month(Date) == mese ),count = n())
+     monthlyByCateg[riga, 9]  <- summarize(subset(datiChicago, Year == anno 
+                                                  & Description == descrizioni[8] & month(Date) == mese ),count = n())
+     monthlyByCateg[riga, 10]  <- summarize(subset(datiChicago, Year == anno
+                                                   & Description == descrizioni[9] & month(Date) == mese ),count = n())
+     monthlyByCateg[riga, 11]  <- summarize(subset(datiChicago, Year == anno 
+                                                   & Description == descrizioni[10] & month(Date) == mese ),count = n())
+     monthlyByCateg[riga, 12]  <- summarize(subset(datiChicago, Year == anno 
+                                                   & Description == descrizioni[11] & month(Date) == mese ),count = n())
+     monthlyByCateg[riga, 13]  <- summarize(subset(datiChicago, Year == anno 
+                                                   & Description == descrizioni[12] & month(Date) == mese ),count = n())
+     mese = mese+1
+     if(riga %% 12 == 0){
+       anno = anno +1
+       mese = 1
+     } 
+      
+ }
> write.csv(monthlyByCateg, file = "monthlyByCateg.csv")
> #rinomino colonna
> names(monthlyByCateg)[names(monthlyByCateg) == 'Months'] <- 'time'
> knitr::kable(summary(monthlyByCateg))
\end{Sinput}
\begin{Soutput}
|   |     time          |  AUTOMOBILE |THEFT/RECOVERY: AUTOMOBILE |TRUCK, BUS, MOTOR HOME |ATT: AUTOMOBILE |THEFT/RECOVERY: TRUCK,BUS,MHOME |CYCLE, SCOOTER, BIKE W-VIN |ATT: TRUCK, BUS, MOTOR HOME |THEFT/RECOVERY: CYCLE, SCOOTER, BIKE NO VIN |ATTEMPT: CYCLE, SCOOTER, BIKE W-VIN |THEFT/RECOVERY: CYCLE, SCOOTER, BIKE W-VIN |CYCLE, SCOOTER, BIKE NO VIN |ATTEMPT: CYCLE, SCOOTER, BIKE NO VIN |
|:--|:------------------|:------------|:--------------------------|:----------------------|:---------------|:-------------------------------|:--------------------------|:---------------------------|:-------------------------------------------|:-----------------------------------|:------------------------------------------|:---------------------------|:------------------------------------|
|   |Min.   :2001-01-01 |Min.   : 550 |Min.   : 31.0              |Min.   :  8.0          |Min.   : 14.0   |Min.   : 0.00                   |Min.   : 0.00              |Min.   : 0.000              |Min.   :0.00000                             |Min.   :0.0000                      |Min.   :0.0000                             |Min.   :0.0000              |Min.   :0.00000                      |
|   |1st Qu.:2005-01-01 |1st Qu.: 897 |1st Qu.: 65.0              |1st Qu.: 37.0          |1st Qu.: 37.0   |1st Qu.: 4.00                   |1st Qu.: 7.00              |1st Qu.: 1.000              |1st Qu.:0.00000                             |1st Qu.:0.0000                      |1st Qu.:0.0000                             |1st Qu.:0.0000              |1st Qu.:0.00000                      |
|   |Median :2009-01-01 |Median :1246 |Median : 86.0              |Median :139.0          |Median : 49.0   |Median :11.00                   |Median :22.00              |Median : 5.000              |Median :0.00000                             |Median :0.0000                      |Median :0.0000                             |Median :0.0000              |Median :0.00000                      |
|   |Mean   :2008-12-30 |Mean   :1191 |Mean   :111.2              |Mean   :125.6          |Mean   : 55.6   |Mean   :14.28                   |Mean   :24.39              |Mean   : 5.927              |Mean   :0.05181                             |Mean   :0.4611                      |Mean   :0.9793                             |Mean   :0.5907              |Mean   :0.03109                      |
|   |3rd Qu.:2013-01-01 |3rd Qu.:1430 |3rd Qu.:153.0              |3rd Qu.:191.0          |3rd Qu.: 66.0   |3rd Qu.:23.00                   |3rd Qu.:37.00              |3rd Qu.:10.000              |3rd Qu.:0.00000                             |3rd Qu.:1.0000                      |3rd Qu.:2.0000                             |3rd Qu.:1.0000              |3rd Qu.:0.00000                      |
|   |Max.   :2017-01-01 |Max.   :2296 |Max.   :372.0              |Max.   :281.0          |Max.   :163.0   |Max.   :40.00                   |Max.   :87.00              |Max.   :20.000              |Max.   :1.00000                             |Max.   :4.0000                      |Max.   :6.0000                             |Max.   :4.0000              |Max.   :1.00000                      |
\end{Soutput}
\begin{Sinput}
> 
\end{Sinput}
\end{Schunk}
A questo punto posso creare una funzione che mi gestisce le varie colonne e mi stampa un summary

\begin{Schunk}
\begin{Sinput}
> analisiDescr <- function(file, colStart, colEnd){
+   dataset <- file[,colStart:colEnd]
+   dim <- length(dataset)
+   #2 significa colonne, #1 righe
+   media <- apply(dataset, 2 , mean, na.rm = TRUE)
+   varianza <-apply(dataset, 2,var, na.rm = TRUE)
+   devStd <- sqrt(varianza)
+   minimo <- apply(dataset, 2, min, na.rm = TRUE)
+   massimo <- apply(dataset, 2, max, na.rm = TRUE)
+   Desc <- data.frame(media, varianza, devStd, minimo, massimo)
+   colnames <- c("media", "varianza", "devStd", "minimo", "massimo")
+   return(Desc)
+ }
> dati <- analisiDescr(monthlyByCateg, 2, 13)
> dati
\end{Sinput}
\begin{Soutput}
                                                   media     varianza
AUTOMOBILE                                  1.190508e+03 1.018440e+05
THEFT/RECOVERY: AUTOMOBILE                  1.111969e+02 4.053024e+03
TRUCK, BUS, MOTOR HOME                      1.256425e+02 6.000668e+03
ATT: AUTOMOBILE                             5.560104e+01 7.423244e+02
THEFT/RECOVERY: TRUCK,BUS,MHOME             1.428497e+01 1.309444e+02
CYCLE, SCOOTER, BIKE W-VIN                  2.438860e+01 3.747701e+02
ATT: TRUCK, BUS, MOTOR HOME                 5.927461e+00 2.306763e+01
THEFT/RECOVERY: CYCLE, SCOOTER, BIKE NO VIN 5.181347e-02 4.938472e-02
ATTEMPT: CYCLE, SCOOTER, BIKE W-VIN         4.611399e-01 5.102008e-01
THEFT/RECOVERY: CYCLE, SCOOTER, BIKE W-VIN  9.792746e-01 1.770402e+00
CYCLE, SCOOTER, BIKE NO VIN                 5.906736e-01 9.930376e-01
ATTEMPT: CYCLE, SCOOTER, BIKE NO VIN        3.108808e-02 3.027850e-02
                                                 devStd minimo massimo
AUTOMOBILE                                  319.1300042    550    2296
THEFT/RECOVERY: AUTOMOBILE                   63.6633610     31     372
TRUCK, BUS, MOTOR HOME                       77.4639813      8     281
ATT: AUTOMOBILE                              27.2456304     14     163
THEFT/RECOVERY: TRUCK,BUS,MHOME              11.4430944      0      40
CYCLE, SCOOTER, BIKE W-VIN                   19.3589793      0      87
ATT: TRUCK, BUS, MOTOR HOME                   4.8028770      0      20
THEFT/RECOVERY: CYCLE, SCOOTER, BIKE NO VIN   0.2222267      0       1
ATTEMPT: CYCLE, SCOOTER, BIKE W-VIN           0.7142834      0       4
THEFT/RECOVERY: CYCLE, SCOOTER, BIKE W-VIN    1.3305644      0       6
CYCLE, SCOOTER, BIKE NO VIN                   0.9965127      0       4
ATTEMPT: CYCLE, SCOOTER, BIKE NO VIN          0.1740072      0       1
\end{Soutput}
\end{Schunk}
Dall'analisi dei dati notiamo che i dati sui furti di automobili (come quelli di mezzi pesanti) vriano notevolmente. Studio quindi la presenza di outliers.
Piccola osservazione: se vi rubano un mezzo a chicago non sperate di recuperarlo, sciocchi.
\begin{Schunk}
\begin{Sinput}
> furti <- boxplot(x= as.list(monthlyByCateg[,2:13]))
\end{Sinput}
\end{Schunk}

Come si evince dal grafico ci sono degli outliers potenziali sopratutto nei grafici 1, 2, 4 e 6 che, mi accingo a rimuovere tramite identify, che fornisce la possiiblità di intervenire direttamente col mouse sul grafico.
\begin{Schunk}
\begin{Sinput}
> furtiAuto <- boxplot(x= monthlyByCateg$AUTOMOBILE)
> darimuovere <- identify(rep(3, length(monthlyByCateg[,2]))+ monthlyByCateg[2])
> boxplot( totPerAnno$Totale.annuo)
> furtiAuto <- boxplot(x= monthlyByCateg$AUTOMOBILE)
> 
\end{Sinput}
\end{Schunk}
Sarò sincero... la funzione "identify" mi ha sempre risposto \texttt{warning: no point within 0.25 inches} nonostante cliccassi correttamente sul punto outlier segnalato, al che ho proceduto alla rimozione manuale dell'outlier, corrispondente alla misurazione su Ottobre 2001.

Al che rimuoviamo gli outliers con la seguente funzione:
\begin{Schunk}
\begin{Sinput}
> outlierDetector(monthlyByCateg, AUTOMOBILE )